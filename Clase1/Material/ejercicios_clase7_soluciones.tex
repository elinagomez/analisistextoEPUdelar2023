% Options for packages loaded elsewhere
\PassOptionsToPackage{unicode}{hyperref}
\PassOptionsToPackage{hyphens}{url}
%
\documentclass[
]{article}
\usepackage{amsmath,amssymb}
\usepackage{iftex}
\ifPDFTeX
  \usepackage[T1]{fontenc}
  \usepackage[utf8]{inputenc}
  \usepackage{textcomp} % provide euro and other symbols
\else % if luatex or xetex
  \usepackage{unicode-math} % this also loads fontspec
  \defaultfontfeatures{Scale=MatchLowercase}
  \defaultfontfeatures[\rmfamily]{Ligatures=TeX,Scale=1}
\fi
\usepackage{lmodern}
\ifPDFTeX\else
  % xetex/luatex font selection
\fi
% Use upquote if available, for straight quotes in verbatim environments
\IfFileExists{upquote.sty}{\usepackage{upquote}}{}
\IfFileExists{microtype.sty}{% use microtype if available
  \usepackage[]{microtype}
  \UseMicrotypeSet[protrusion]{basicmath} % disable protrusion for tt fonts
}{}
\makeatletter
\@ifundefined{KOMAClassName}{% if non-KOMA class
  \IfFileExists{parskip.sty}{%
    \usepackage{parskip}
  }{% else
    \setlength{\parindent}{0pt}
    \setlength{\parskip}{6pt plus 2pt minus 1pt}}
}{% if KOMA class
  \KOMAoptions{parskip=half}}
\makeatother
\usepackage{xcolor}
\usepackage[margin=1in]{geometry}
\usepackage{longtable,booktabs,array}
\usepackage{calc} % for calculating minipage widths
% Correct order of tables after \paragraph or \subparagraph
\usepackage{etoolbox}
\makeatletter
\patchcmd\longtable{\par}{\if@noskipsec\mbox{}\fi\par}{}{}
\makeatother
% Allow footnotes in longtable head/foot
\IfFileExists{footnotehyper.sty}{\usepackage{footnotehyper}}{\usepackage{footnote}}
\makesavenoteenv{longtable}
\usepackage{graphicx}
\makeatletter
\def\maxwidth{\ifdim\Gin@nat@width>\linewidth\linewidth\else\Gin@nat@width\fi}
\def\maxheight{\ifdim\Gin@nat@height>\textheight\textheight\else\Gin@nat@height\fi}
\makeatother
% Scale images if necessary, so that they will not overflow the page
% margins by default, and it is still possible to overwrite the defaults
% using explicit options in \includegraphics[width, height, ...]{}
\setkeys{Gin}{width=\maxwidth,height=\maxheight,keepaspectratio}
% Set default figure placement to htbp
\makeatletter
\def\fps@figure{htbp}
\makeatother
\setlength{\emergencystretch}{3em} % prevent overfull lines
\providecommand{\tightlist}{%
  \setlength{\itemsep}{0pt}\setlength{\parskip}{0pt}}
\setcounter{secnumdepth}{-\maxdimen} % remove section numbering
\ifLuaTeX
  \usepackage{selnolig}  % disable illegal ligatures
\fi
\IfFileExists{bookmark.sty}{\usepackage{bookmark}}{\usepackage{hyperref}}
\IfFileExists{xurl.sty}{\usepackage{xurl}}{} % add URL line breaks if available
\urlstyle{same}
\hypersetup{
  pdftitle={Ejercicios Clase 7},
  hidelinks,
  pdfcreator={LaTeX via pandoc}}

\title{Ejercicios Clase 7}
\author{}
\date{\vspace{-2.5em}08/11/2022}

\begin{document}
\maketitle

header-includes: -

\usepackage{bbm}

\hypertarget{introducciuxf3n}{%
\section{Introducción}\label{introducciuxf3n}}

El presente \textbf{documento} contiene algunos resultados del
procesamiento de menciones parlamentarias realizadas en el marco del
curso de \emph{R aplicado al análisis cualitativo}.

\begin{longtable}[]{@{}lr@{}}
\caption{Palabras más mencionadas}\tabularnewline
\toprule\noalign{}
& N \\
\midrule\noalign{}
\endfirsthead
\toprule\noalign{}
& N \\
\midrule\noalign{}
\endhead
\bottomrule\noalign{}
\endlastfoot
pol & 3808 \\
est & 3704 \\
ley & 2945 \\
tambi & 2846 \\
ser & 2718 \\
mujeres & 2504 \\
proyecto & 2419 \\
hoy & 2227 \\
tica & 2217 \\
derechos & 2161 \\
\end{longtable}

\newpage

\hypertarget{nubes-de-palabras}{%
\section{Nubes de palabras}\label{nubes-de-palabras}}

\hypertarget{general}{%
\subsection{General}\label{general}}

A continuación hago una nube de palabras general y por grupos.

\begin{figure}
\includegraphics[width=0.8\linewidth]{ejercicios_clase7_soluciones_files/figure-latex/unnamed-chunk-4-1} \caption{Nube}\label{fig:unnamed-chunk-4}
\end{figure}

\#\#Grupos

\begin{figure}
\includegraphics[width=1\linewidth]{ejercicios_clase7_soluciones_files/figure-latex/unnamed-chunk-5-1} \caption{Nube de palabras por grupos}\label{fig:unnamed-chunk-5}
\end{figure}

\end{document}
